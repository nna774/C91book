\documentclass[12pt,b6book]{book}

\usepackage{xltxtra}
\setmainfont[Ligatures=TeX]{IPAPMincho}
\setsansfont{IPAPGothic}
\setmonofont{IPAGothic}
\XeTeXlinebreaklocale "ja"
\usepackage{graphicx}
%\usepackage[unicode]{hyperref}
\usepackage{hyperref}
\usepackage{pdfpages}

\usepackage{listings}
\lstset{%
  backgroundcolor={\color[gray]{.85}},%
  basicstyle={\small},%
  frame={tb},
  breaklines=true,
  columns=[l]{fullflexible},%
%  lineskip=-0.5ex%
}
\usepackage{verbatim}

\usepackage{ulem}

% myjapanese.sty
\RequirePackage{xltxtra}
\def\en#1{{\engrm#1}} % 欧文フォントで出力
% 日本語を含む段落を行分割するための設定
\XeTeXlinebreaklocale ``ja''
\XeTeXlinebreakskip=0pt plus 1pt minus 0.1pt
\XeTeXlinebreakpenalty=0
% 半角分戻る
\def\<{\@ifstar{\zx@hwback\nobreak}{\zx@hwback\relax}}
\def\zx@hwback#1{\leavevmode#1\hskip-.5em\relax}
% 簡易レイアウト設定
\RequirePackage[scale=0.8, margin=1in]{geometry}
%\RequirePackage{indentfirst}
\RequirePackage{setspace}
\setstretch{1}
\parindent=1em

\begin{document}

\frontmatter
%% 表紙
%% \pagenumbering{Alph}
%%\insertpage{images/KOKORO100.pdf}
%%%%%%%%%%%%%%\includepdf{images/hyoushi.pdf}
%%\includegraphics[width=\paperwidth]{images/KOKORO100.png}
%% \addtocounter{ptc}{1} % nande-
\tableofcontents
%% \newpage
%% 
%% \newpage

\mainmatter

%% \input{nona7}

%% \input{murata}

\chapter*{あとがきの国}

\paragraph{NoNameA 774}

{夏休み終了ぐらいから体調を崩していてギリギリまで出すか悩みましたが、なんとか形になりました。
読んでくださってありがとうございます……。

今回murataくんの記事とかは色についての言及が含まれているのに、カラーで刷れなかったので残念です。
是非PDFで見てみてください。}

\paragraph{murata}

{Pietの基礎知識を前提として書いてしまった感があるのですが頑張って解読してくれたら幸いです。}

\paragraph{著作権表示}

{\small この本を許可無くスキャン等してインターネット等に公開することを禁じます。
インターネットに公開する際は、\url{https://nna774.net/piet/C91Book.pdf}よりid:pass piet:editorでダウンロードしてから公開してください。}

{\footnotesize より正確に言うと、Creative Commons BY-SA 4.0に従うか、GNU Free Document License 1.3 or any later versionに従うことで、あなたはこの本を自由に共有、頒布等をすることができます。
  今回印刷の際にモノクロで印刷となってしまったので、是非カラーのPDFをご覧ください……。
しばらくは私からは認証等無しにこの本を公開する予定はありませんが(購入して頂いた人が特権的に読めるように)、もし十分広くインターネットでこの本が入手できるようになった場合には私のWebページで公開します。また、「これ以上有料での頒布は行なわない」となった際などにも無償で公開するかもしれません。}


\null\vfill
\section*{奥付}
\hrule\vskip.5mm\hrule\vskip3mm
\begin{tabular}{ll}
  2016/12/29 & 初版発行\\
  hash: & \href{https://github.com/nna774/C89Book2/tree/3e0c955600ecae2e6549ea95b9c9eee7587ef01c}{3e0c955600ecae2e6549ea95b9c9eee7587ef01c}(ココの変更)\\
  著作・発行 & NoNameA 774 (nonamea774@nnn77) \\
  サークル & いっと☆わーくす! \\
  メールアドレス & \href{mailto:nonamea774@gmail.com}{\nolinkurl{nonamea774@gmail.com}}\\
  Web & \url{https://nna774.net/}\\
  Twitter & @nonamea774\\
  GPG Key & 0x0C3E3AB2\\
  fingerprint & 674A 287A 21D2 2431 AD8F \\
  & D328 AEF3 C3C7 0C3E 3AB2\\
  Keybase.io & \url{https://keybase.io/nona}\\
\end{tabular}
\vskip3mm\hrule\vskip3mm

コメント

%% \newpage
%%  
%% \newpage
%%%%%%%%%%%%%%%%%%%%%%%%%%%%%%%%%%\includepdf{images/KOKORO100_kaisetsu.pdf}

\end{document}
