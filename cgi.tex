\chapter[KMCのエイプリルpiet]{KMCのエイプリルpiet}

KMCは今年のエイプリルフールでpietでwebサーバを書いて(描いて?)いました。
4/1には\url{http://april-2016.kmc.gr.jp}\footnote{今はこのURIにアクセスすると、解説ブログへとリダイレクトされます。是非一度ご覧ください。}でpietで書かれたHTTP serverが動いていて、
それがnginxの下にぶらさがっていたようです\footnote{ようです、というのは、これには私はかかわっていないからです(ここで気付きましたが、この記事はのなの二つ目の記事です。はじめにが無いので判り辛いですね)。}。

使われていたサーバは以下のようなソースコードとなっていました。……とここにソースコードの画像をほどんど線のように見えるだろうなと予想しながらも貼ろうとしましたが、
57286 x 153 の大きさを持ち、\TeX の制限? で貼ることができませんでした。
\url{https://raw.githubusercontent.com/kmc-jp/2016-aplil-fool/master/htmlserver.png}\footnote{\url{https://github.com/kmc-jp/2016-aplil-fool/}にレポジトリがあり、その中にこのコードも含まれています。}
で見ることができます。

巨大すぎるので、最初のほうを切り出してきますと、このような感じです。
\centerimage{images/outx10.png}
これだけでpietの自動生成に詳しい人は、だいたいどのように書かれたのか予想がつくかもしれません。
\url{http://www.toothycat.net/wiki/wiki.pl?MoonShadow/Piet}などが生成するpietに似ていますね。
